\chapter*{Abstract}

This report documents the development of an \textit{animal classification AI} using a more \textit{old-school approach} of Visual-Bag-of-Words models.
This AI is capable of differentiating 12 different animals.
These models, and thus the AI, are developed in Python-based Jupyter Notebooks accompanied by this document.
This animal classification AI was developed as a fulfilment of the Machine Learning course requirements and was used to compete in the organised Kaggle competition \citep{kaggle_competition}.

Part \ref{part:about_the_code} of this report discusses the accompanied code in general.
Section \ref{section:inc_files} explains which files are the most important. 
Section \ref{section:ideology_dev_code} describes the ideology used to created the code. 
To make testing multiple models easier, \textit{a template} for model exploration was created and is discussed in section \ref{section:typical_model_exploration}.

In part \ref{part:data_analysis}, the data analysis part of this project is discussed.
Section \ref{section:DA_data_distribution} talks about the unbalanced data distribution.
In the next section, section \ref{section:DA_deeper_look_data}, a deeper look is taken into the data and possible preprocessing is discussed.
The last few sections of this part discuss how the feature extraction is dealt with and what the numerical representation looks like.

The linear baseline model is discussed in part \ref{part:linear_baseline}.
This model is a fine-tuned Logistic Regression model from the SciKit Learn library.
This model is often used to compare other models with.
Only models that perform better then this baseline model should be considered.
This part discusses the parameters used and the road to finding those optimal parameters.

TODO XXX

Finally, don't let the page count scare you.
Due to clear separation in parts and large figures the page count of this document doesn't represent the length of it's actual content.
Content that isn't crucial or is a repetition of supplied information is given as appendixes.