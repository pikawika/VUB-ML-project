% TODO volledig
\part{Conclusion}
\label{part:conclusion}

%------------------------------------

\section{What is learned and achieved}
\label{section:con_achieved}

This project has given a lot of insight into the working of different models and how model exploration is done.
It has also shown that just blindly trusting \texttt{GridSearchCV} or received validation scores isn't a good idea.
Whilst there is still a lot of room for improvement, the resulting final model seems to be pretty balanced and has an average accuracy of 50\% or more for all classes.
Taking into account this is using the more \textit{old-school} Visual-Bag-of-Words approach, the results are satisfying and above average when looking at the public Kaggle leaderboard.

%------------------------------------

\section{The hurdles}
\label{section:con_hurdles}

Some of the calculations can take a lot of time, which means making small errors can be very time punishing.
Some of the fundamentals of a model, such as used descriptor and cluster amount, are also fixed for the whole model exploration.
Changing these fundamental parameters can influence the whole model's performance and thus can not be done \textit{after the fact}.
This resulted in much recomputing work, which again, costs a lot of time.
Due to limited computation power and time, experimenting with descriptors had to be kept to a minimum. 
This is most likely the part where most of the remaining gains can be made.
There is one known \textit{data leakage} present, namely the clustering of the features is done overall training images instead of only the training split.
When this flaw was discovered, the K-Means clustering was already in use.
Since this clustering algorithm takes 12+ hours this type of \textit{data leakage} is kept in.
This also only has an effect on the validation set (not the actual test set) and is expected to be minimal, thus it is not as bad as it sounds.

%------------------------------------

\section{Further extensions}
\label{section:con_extensions}

As discussed, the models created in this report are by no means perfect.
Most of the interesting further extensions are in experimenting with different descriptors and settings for them.
A list of these possible extensions is given in section \ref{section:opt_unexplored}.