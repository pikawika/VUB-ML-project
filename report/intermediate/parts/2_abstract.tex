\chapter*{Abstract}

This intermediate report documents the development of an animal classification AI using a more "old-school" approach of Visual-Bag-of-Words models.
The AI should be capable of differentiating 12 different animals.
These models, and thus the AI, are developed in Python-based Jupyter Notebooks accompanied by this document.
This animal classification AI was developed as a fulfilment  of the Machine Learning course requirements and was used to compete in the organised Kaggle competition \citep{kaggle_competition}.

Part \ref{part:about_the_code} of this report discusses the accompanied code in general.
Section \ref{section:inc_files} explains which files are accompanied and which are the most important. 
Section \ref{section:ideology_dev_code} describes the ideology used to created the code. 
To make testing multiple models easier, a form of \emph{pipeline} was created and is discussed in section \ref{section:typical_model_exploration}.

In part \ref{part:data_analysis}, the data analysis part of this project is discussed.
Section \ref{section:DA_data_distribution} talks about the unbalanced data distribution.
In the next section, section \ref{section:DA_deeper_look_data}, a deeper look is taken into the data and possible preprocessing is discussed.
The last 2 sections of this part, \ref{section:DA_feature_extraction} and \ref{section:DA_numerical_representation}, discuss how the feature extraction is dealt with and what the numerical representation looks like.

The linear baseline model is discussed in part \ref{part:linear_baseline}.
This model is a fine-tuned Logistic Regression model from the SciKit Learn library.
This model is often used to compare other models with.
Only models that perform better then this baseline model should be considered.
This part discusses the parameters used and the road to getting those optimal parameters.

Finally, part \ref{part:whats_next} discusses future plans for this project.
Section \ref{section:further_development} lists possible topics that can be explored to create a better animal classification AI.
In the last section, section \ref{section:open_issues}, some open issues are discussed.